\documentclass{statement}

% Author:  Lorena Barba


\setromanfont[Mapping=tex-text,
                 SmallCapsFont={Palatino},
                 SmallCapsFeatures={Scale=0.85}]{Palatino}
\setsansfont[Mapping=tex-text,Scale=0.9]{Optima} 
\setmonofont[Mapping=tex-text,Scale=0.85]{Monaco}
\newfontfamily\titfont[Mapping=tex-text]{Source Sans Pro ExtraLight}
\newfontfamily\secfont[Mapping=tex-text]{Source Sans Pro Bold}
\newfontfamily\subsecfont[Mapping=tex-text]{Source Sans Pro}
\renewcommand{\captionlabelfont}{\bf\sffamily}
\lhead{}
\chead{}
\rhead{\titfont Terminologies for Reproducible Research}
\lfoot{}
\cfoot{\thepage}
\rfoot{}

%\usepackage[T1]{fontenc}
%\usepackage[sc]{mathpazo}   % option clash with wrapfig !!

% Define the color to use in links:
\definecolor{linkcol}{rgb}{0.278,0.541,0.459}%

\definecolor{sectcol}{rgb}{0.63,0.16,0.16}
\definecolor{palevioletred4}{rgb}{0.545,0.278,0.365}
\definecolor{gray40}{rgb}{0.40,0.40,0.40}
\definecolor{gray26}{rgb}{0.26,0.26,0.26}
\definecolor{olive}{rgb}{0.5,0.5,0.0}
\definecolor{gray78}{cmyk}{0,0,0,0.22}



\usepackage[
    xetex,
    pdftitle={Terminologies for Reproducible Research},
    pdfauthor={Lorena Barba},
    pdfpagemode={UseOutlines},
    pdfpagelayout={TwoColumnRight},
    bookmarks, bookmarksopen,bookmarksnumbered={True},
    pdfstartview={FitH},
    colorlinks, linkcolor={linkcol},citecolor={linkcol},urlcolor={linkcol}
    ]{hyperref}

%% Define a new style for the url package that will use a smaller font.
\makeatletter
\def\url@leostyle{%
  \@ifundefined{selectfont}{\def\UrlFont{\sf}}{\def\UrlFont{\small\ttfamily}}}
\makeatother
%% Now actually use the newly defined style.
\urlstyle{leo}

% this handles hanging indents for publications
\def\rrr#1\\{\par
\medskip\hbox{\vbox{\parindent=2em\hsize=6.12in
\hangindent=4em\hangafter=1#1}}}

\def\baselinestretch{1}
\setlength{\parindent}{0mm} \setlength{\parskip}{0.8em}

\newlength{\up}
\setlength{\up}{-4mm}

\newlength{\hup}
\setlength{\hup}{-2mm}

\sectionfont{\secfont \color{palevioletred4}\vspace{-2mm}}
\subsectionfont{\secfont\vspace{-4mm}}
\subsubsectionfont{\normalsize\mdseries\itshape\vspace{-4mm}} %\itshape
\paragraphfont{\bfseries}

\newfontfamily\notefont[Mapping=tex-text,Scale=1]{Optima}

% -----------------------------------------------------------------------------------------------------------
\begin{document}
% -----------------------------------------------------------------------------------------------------------

% reset page numbering to 1
\pagenumbering{arabic}
\renewcommand{\thepage} {\arabic{page}}

\thispagestyle{empty}

{\titfont \Huge Terminologies for Reproducible Research} 
\medskip

{\fontspec[Mapping=tex-text,Scale=0.8]{Source Sans Pro} Lorena A. Barba, the George Washington University, Washington D.C. 

January 2018 --- UNFINISHED DRAFT}



\vspace{1cm}

\section*{Introduction}
\vspace{\up}

Reproducible research---by its many names---has come to be regarded as a key concern across disciplines and stakeholder groups. Funding agencies and journals, professional societies and even mass media are paying attention, often focusing on the so-called ``crisis'' of reproducibility. One big problem keeps coming up among those seeking to tackle the issue: different groups are using terminologies in utter contradiction with each other. 
In July 2017, over a dozen participants joined the \href{https://collegeville.github.io/repeto/ReproducibilityWorkshop2017.html}{Workshop on Reproducibility Taxonomies for Computing and Computational Science} at the National Science Foundation, to appraise the variety of terminologies.
My presentation at that event \cite[]{barba2017-repeto} condensed a catalog of uses of the recurrent terms \emph{reproduce} and \emph{replicate}, often meaning different things but sometimes interchangeable.

Looking at a broad sample of publications in different fields, we can classify their terminology via decision tree: they either, \emph{A}---make no distinction between the words \emph{reproduce} and \emph{replicate}, or \emph{B}---use them distinctly.  
If \emph{B}, then they are commonly divided in two camps. 
In a spectrum of concerns that starts at a minimum standard of ``same data$+$same methods$=$same results,'' to ``new data and/or new methods in an independent study$=$same findings,'' group $1$ calls the minimum standard \emph{reproduce}, while group $2$ calls it \emph{replicate}.
This direct swap of the two terms aggravates an already weighty issue.
By attempting to inventory the terminologies across disciplines, I hope that some patterns will emerge to help us resolve the contradictions.


\section*{Pioneers of reproducible research}
\vspace{\up}

The first appearance of the phrase ``reproducible research'' in a scholarly publication appears to be an invited paper presented at the 1992 meeting of the Society of Exploration Geophysics (SEG), from the group of Jon Claerbout at Stanford \cite[]{claerbout1992}.
Claerbout pioneered the use of computers in processing and filtering seismic exploration data \cite[]{wiki:claerbout}.
From at least 1990, he required his students' PhD theses to conform to a standard of reproducibility.
His idea of reproducible research was to leave finished work (an article or a thesis) in a state that allowed colleagues to reproduce the calculation, analysis and final figures by executing a single command.
The goal was to merge a publication with its underlying computational analysis.
They used an automation tool called \texttt{make}, which builds software from source code by reading through a \texttt{makefile}: a list of commands to be executed in sequence.
Their workflow combined a set of standardized commands (burn, build, view, clean), and a filing system for the research compendium associated with the paper (data sets, programs, scripts, parameter files, makefiles).
Explicitly, \cite{claerbout1992} provide a ``definition of reproducibility in computationally oriented research.''
The original SEG paper is somewhat dated, but the group presented an updated overview in \cite{schwabETal2000}.
Here, the authors emphasize the limitations of the traditional methods of scientific publication, especially for computational research. 
In their vision of reproducible research, readers should be able to rebuild published results ``using the author's underlying programs and raw data.''
Implicitly, they are advocating for open code and data.

At Stanford, statistics professor David Donoho learned of Claerbout's methods in the early 1990s, and began adopting (and later promoting) them. 
In \cite{buckheit_donoho1995}, the authors state that``reproducibility of experiments in seismic exploration requires having the complete software environment available in other laboratories and the full source code available for inspection, modification, and application under varied parameter settings.'' 
This is because the goal in this field is to produce an image of the earth's sub-surface, from the datasets of seismic exploration signals. The software generating the image is key to the final results. 
And here is where the authors write the often-quoted ``slogan'' of reproducible research (paraphrasing Claerbout): 
\begin{quote}
\emph{``An article about computational science in a scientific publication is not the scholarship itself, it is merely advertising of the scholarship. The actual scholarship is the complete software development environment and the complete set of instructions which generated the figures.''}
\end{quote}

\cite{buckheit_donoho1995} include many technical details that seem dated now, so perhaps a better place to read about this group's thinking and practice is \cite{donohoETal2009}. 
This appears to be the first article to publicly state that reproducibility depends on open code and data. The authors define \emph{reproducible computational research} as that ``in which all details of computations---code and data---are made conveniently available to others.'' 
They used the now-recurrent charged language about a ``credibility crisis'' in computational science, and worry that computation cannot claim to be the ``third branch'' of science because most computational results cannot be verified. 
In the two traditional branches, standards of practice already exist for managing the ubiquity of error: deductive science uses formal logic and the mathematical proof, while empirical science uses statistical hypothesis testing and detailed methods reporting. 
The authors also counter the view that reproducibility means re-implementing the research software from the ground up. 
Because, in that case, if the new results did not match, ``the only way we'd ever get to the bottom of such a discrepancy is if we both worked reproducibly and studied detailed differences between code and data.''

The pioneering efforts of Claerbout and Donoho influenced many others. But their concerns centered on research involving computational analysis of unique data (recordings of seismic waves). 
They did not deal with the situation where another researcher might collect \emph{new} data, re-doing a study or an experiment following an original design, then analyze this data to compare the final findings with the original. 
Situations like this arise in many empirical fields, where scientific findings must be confirmed by independent studies. 
\cite{pengETal2006} distinguish the term \emph{replication} for this scenario. They say:
``Scientific evidence is strengthened when important findings are \emph{replicated} by multiple independent investigators using independent data, analytical methods, laboratories, and instruments.'' 
But epidemiologic studies are expensive, time-consuming, and often impossible to replicate (e.g., when they rely on opportunistic data, say, from an infectious outbreak). 
``An attainable minimum standard is \emph{reproducibility}, which calls for data sets and software to be made available for verifying published findings and conducting alternative analyses,'' they say.

In the social sciences, Harvard professor Gary King, one of the most cited political scientists of this generation, also pioneered the reproducibility concerns. 
In \cite{king1995}, he writes: ``The replication standard holds that sufficient information exists with which to understand, evaluate, and build upon a prior work if a third party could replicate the results without any additional information from the author.'' He uses the term \emph{replication} throughout, not making a distinction like we have in the Claerbout/Donoho/Peng writings on reproducible research. 
In fact, his paper uses ``replication'' 96 times, ``replicate'' 21 times, and ``reproduce'' (as verb) only twice. E.g., ``good science requires that we be able to reproduce existing numerical results\ldots''
A more recent work co-authored by him \cite[]{lazerETal2014}, still does not use the words ``reproduce/reproducibility/reproducible,'' and only uses ``replicate.'' 
This appears to be the common usage in his field, as a Google Scholar search with ``replication political science`` gives 339,000 results, and with ``reproducible political science'' gives just 59,700 results (checked 01/14/2017).


\section*{Conflicting terminologies}
\vspace{\up}

As mentioned above, Claerbout and Donoho's pioneering work influenced many others, who adopt their definition of reproducible research. 
For example, \cite{gentlemanETal2007}: ``By reproducible research, we mean research papers with accompanying software tools that allow the reader to directly reproduce the results and employ the computational methods that are presented in the research paper.''
\cite{vandewalleETal2009}: ``A research work is called reproducible if all information relevant to the work, including, but not limited to, text, data and code, is made available, such that an independent researcher can reproduce the results.''
\cite{leveque2009}: ``The idea of `reproducible research' in scientific computing is to archive and make publicly available all the codes used to create a paper's figures or tables, preferably in such a manner that readers can download the codes and run them to reproduce the results.''
The distinction with the idea of replication, introduced by Peng, also appears in other works. 
The \emph{Annals of Internal Medicine} issued a statement saying: ``Independent replication by independent scientists in independent settings provides the best assurance that a scientific finding is valid; however, the resources and time required for high-quality clinical studies makes literal replication of published studies a slow corrective to any errors in the original publication. However, scientists and journal editors can promote `reproducible research,' [which ensures] that independent scientists can reproduce published results by using the same procedures and data as the original investigators'' \cite[]{laineETal2007}. 
In an editorial, the \emph{Int. Journal of Forecasting} announces that it will publish \emph{replication studies}: works attempting to independently verify research findings ``under the same or very similar conditions'' \cite[]{hyndman2010}. 
\emph{Science} published a special issue on ``Data Replication \& Reproducibility'' in 2011, to which the introduction reads: ``Replication---the confirmation of results and conclusions from one study obtained independently in another---is considered the scientific gold standard'' \cite[]{jasnyETal2011}.
Peng's own (widely cited) article in this special issue introduces the idea of a \emph{reproducibility spectrum}, in which reproducible research is a ``minimum standard for judging scientific claims when full independent replication of a study is not possible.'' \cite[]{peng2011}

On the basis of the multiple references cited above, here are concise definitions that convey the meanings of the Claerbout/Donoho/Peng convention:

\begin{description}
\item[Reproducible research:] Authors provide all the necessary data and the computer codes to run the analysis again, re-creating the results.
\item[Replication:] A study that arrives at the same scientific findings as another study, collecting new data (possibly with different methods) and completing new analyses.
\end{description}

Unfortunately, conflicting terminologies started to spread in recent years. 
It appears that the first work to directly swap the usage of \emph{reproducible} and \emph{replicable} is \cite{drummond2009}. 
This is a workshop contribution to a conference in the field of machine learning (meaning, it received light or no peer review). 
The author seems to arbitrarily swap existing terminology; he says: ``I have used the term replicability for what others have called reproducibility in our literature.''
He provides no more justification than ``I think it reasonable.''
Mark Liberman, Distinguished Professor of Linguistics at the University of Pennsylvania, analyzed the usage of terms in the literature, and referring to Drummond's paper he concluded: ``Since the technical term `reproducible research' has been in use since 1990, and the technical distinction between reproducible and replicable at least since 2006, we should reject [the] attempt to re-coin technical terms reproducible and replicable in senses that assign the terms to concepts nearly opposite to those used in the definitions by Claerbout, Peng and others.'' \cite[]{liberman2015}

Alas, before the scrutiny of the linguistics professor, others picked up on the swapped terminology. 
\cite{casadevall_fang2010}, citing Drummond, adopt it and discuss various scenarios in the fields of microbiology and immunology. 
Other examples include \cite{davison2012}, \cite{loscalzo2012}, \cite{crook2013}, \cite{cooper2015}, among many others. 
Drummond's 2009 paper has 134 citations in Google Scholar (checked Jan.\ 15, 2018), although some are for other reasons \cite[]{dewinter_happee2013} or to plainly point out the discrepant terminology \cite[]{boylan2015}.

Some other works adopted the swapped terms without a plain link to \cite{drummond2009}. 
In \cite{levequeETal2012}, referring to topics of discussion at a July 2011 workshop titled ``Reproducible Research: Tools and Strategies for Scientific Computing,'' the authors write: ``As an example of the lack of a common nomenclature, two sequential speakers provided opposite definitions for replicable and reproducible. (We believe the  first refers to the ability to run a code and produce exactly the same results as published, and the second refers to the ability to create a code that independently verifies the published results using the information provided.)'' 
A citation follows to \cite{stodden2011}, where we read: ``Replication, using author-provided code and data, and independent reproduction work hand-in-hand. We can reserve the term `replicability' for the regeneration of published results from author-provided code and data.'' Followed by an immediate citation to \cite{king1995}, maybe this was prompted by the similarity between King's conditions for a ``replication standard'' and the Claerbout/Donoho concept of reproducible research. 
(Although, as mentioned above, King simply uses ``replication'' for everything.) 
Also in \cite{levequeETal2012}, the practice of ``private reproducibility'' is explained as being able to rebuild own past research results from the precise version of the code that was used to create them. 
In this sense, ``reproducibility'' is that minimum standard in the spectrum, in contrast to ``replicability,'' as Drummond would have it. 
The opposing terminologies coexist in this paper.

A more recent use of terminology in contradiction with the Claerbout/Donoho/Peng convention appears in the effort of the Association of Computing Machinery (ACM) on Result and Artifact Review and Badging.\footnote{ Approved June 2016, \url{https://www.acm.org/publications/policies/artifact-review-badging}}  It deploys across ACM publications a badging system for articles complying with various standards of code and data sharing. With that purpose, it defines the following terminology:
\vspace{\up}
\begin{description}
\item[Repeatability ]  (Same team, same experimental setup.)
The measurement can be obtained with stated precision by the same team using the same measurement procedure, the same measuring system, under the same operating conditions, in the same location on multiple trials. For computational experiments, this means that a researcher can reliably repeat her own computation.
\item[Replicability ] (Different team, same experimental setup.)
The measurement can be obtained with stated precision by a different team using the same measurement procedure, the same measuring system, under the same operating conditions, in the same or a different location on multiple trials. For computational experiments, this means that an independent group can obtain the same result using the author?s own artifacts.
\item[Reproducibility] (Different team, different experimental setup.)
The measurement can be obtained with stated precision by a different team, a different measuring system, in a different location on multiple trials. For computational experiments, this means that an independent group can obtain the same result using artifacts which they develop completely independently.
\end{description}


The source cited for the definitions adopted by the ACM is the International Vocabulary of Metrology \cite[]{jcgm2008}, which establishes terminology for physical measurements. 
Relevant passages from this source are (abridged):
\begin{description}
\item[repeatability condition of measurement]\ldots same measurement procedure, same operators, same measuring system, same operating conditions and same location, and replicate measurements on the same or similar objects over a short period of time (p.23);
\item[reproducibility condition of measurement]\ldots set of conditions that includes different locations, operators, measuring systems, and replicate measurements on the same or similar objects (p.24).
\end{description}

The document contains no definition of ``replicability'' but both definitions above use the phrase `replicate measurements.' The scenario for these definitions is a physical quantity being measured, and the precision of that measurement. Repeatability is the precision over successive measurements of the same quantity, with everything kept the same (even the operator), over a short period of time. 
Because measuring tools and procedures have inherent errors, it's relevant to document repeatability in the form of standard deviation or other dispersion characteristics (e.g., graphical).
Reproducibility of measurements involves changing at least one condition, e.g., the instrument, the location, or the operator. 
Because the \emph{same} physical quantity is being measured (``on the same or similar object''), reproducibility of measurement is also expressed by a dispersion statistic, like standard deviation \cite[p.14]{taylorETal1994}. 

Along the way of creating the ACM terminology, based on the metrology document, some leap occurred. Replicability is not defined in \cite{jcgm2008}, and reproducibility could just as well be mapped to the minimum standard of the Peng spectrum, where certainly the operator (researcher) changes, the instrument (computer) does as well, and other conditions may also change, but the object of study remains the same.
Arguably, conditions for physical measurements are a distant analogy for the complex processes of a full scientific workflow. 

\section*{Cataloguing the reproducibility literature}
\vspace{\up}

The conflicting terminologies are at least an annoyance, and at worst an impediment to the progress of science. Yet no solution is at hand beyond a general good practice of always defining the terms used in any particular writing. 
Here is a decision tree to catalogue the terminologies in the literature: 
authors either, $A$---make no distinction between the words \emph{reproduce} and \emph{replicate}, or $B$---use them distinctly.  
If $B$, then they are commonly divided in two camps. 
In a spectrum of concerns that starts at a minimum standard of ``same data$+$same methods$=$same results,'' to ``new data and/or new methods in an independent study$=$same findings,'' group $1$ calls the minimum standard \emph{reproduce}, while group $2$ calls it \emph{replicate}.
$A$ includes \cite{king1995}.
$B1$ corresponds to the Claerbout/Donoho/Peng convention, while $B2$ agrees with \cite{drummond2009} and the ACM terminology.
Table \ref{repro-table} classifies into these groups all the references cited above, and more.

\bigskip

\begin{table}[h]
\caption{Catalogue of terminologies in the literature, with Google Scholar citations (checked Jan.\ 20, 2018).}
\begin{footnotesize}
\begin{tabular}{c c c}
$A$				&	$B1$				&	$B2$ \\ \hline
\cite{king1995}, 527	& \cite{pengETal2006}, 177	& \cite{drummond2009}, 135 \\
\cite{jcgm2008}, 32	& \cite{gentlemanETal2007}, 216	& \cite{casadevall_fang2010}, 58\\
				& \cite{laineETal2007}, 134	& \cite{stodden2011}, 30\\
\cite{dewaldETal1986}, 506 & \cite{vandewalleETal2009}, 266& \cite{davison2012}, 80 \\
\cite{pesaran2003}, 12 & \cite{leveque2009}, 32	& \cite{loscalzo2012}, 31\\
\cite{mccullough2008}, 93 & \cite{hyndman2010}, 20	& \cite{levequeETal2012}, 74\\
\cite{garijoETal2013}, 52	& \cite{jasnyETal2011}, 180& \cite{crook2013}, 16\\
\cite{openscience2012}, 300	& \cite{peng2011}, 552 	& \cite{cooper2015}, 26\\
\cite{openscience2015}, 1573	& 					& \\
\cite{stodden2015}, 19 & \cite{koenkerETal2009}, 58	& \cite{cartwright1991}, 81\\
\cite{duvendackETal2017}, 13 &\cite{delescluseETal2012}, 22	&\\
				&  \cite{sandveETal2013}, 227	& \\
				& \cite{topalidou2015}, 14	& \\
				&\cite{iqbalETal2016}, 67	& \\
				&\cite{stevens2017}, 1	& \\
				& \cite{bollenETal2015}, 12	&\\
\end{tabular}
\end{footnotesize}
\label{repro-table}
\end{table}%

\subsection*{Added references and commentary}
\vspace{\up}

Table \ref{repro-table} includes additional references to those cited in the narrative above.  Quotes from these and/or commentary to justify their grouping follow below.

\cite{koenkerETal2009} cite the Claerbout `slogan,' from \cite{buckheit_donoho1995}, and list various software tools and techniques to enable reproducible research. 
They then describe two `replication case studies,' i.e., efforts to obtain the same results that appeared in other publications. 
In the conclusion, they say: ``the real challenge of reproducible econometric research lies in restructuring incentives to encourage better archiving and distribution of the gory details of computationally oriented research.''
Despite absence of explicit definitions, the usage agrees with group $B1$. 
Similarly, \cite{delescluseETal2012} cite Claerbout, Donoho, and Peng (among many others), and adopt the $B1$ terminology. 
The authors describe their preferred tools for reproducible data analysis of neurophysiological data, and illustrate the process with a detailed example of implementation.

The work by \cite{sandveETal2013} is also in group $B1$. It says: ``As full replication of studies on independently collected data is often not feasible, [\ldots] reproducible research [is] an attainable minimum standard for assessing the value of scientific claims.''
The authors use the phrase `replication studies' several times,  and refer to `inability to replicate findings' and `replication in studies with different data.'
Thus, although they included no explicit terminology definition, their usage agrees with  Claerbout/Donoho/Peng.

\cite{garijoETal2013} lacks any definitions, and does not seem to make a distinction between \emph{replicate} and \emph{reproduce}. However, the first term is used only 3 times, while terms with the root `reproduc-' appear 161 times. The paper deals with reproducing another group's published work using both artifacts provided by original authors and others.
It's listed in group $A$ as a conservative choice, but it could be argued to belong in $B1$.

\cite{topalidou2015} describe an undertaking to reproduce published results that failed initially when the code provided by the original authors did not compile. 
The effort resulted in a collaboration with the original authors, and a reimplementation of the code from scratch, ``following the principles of reproducible computational science as proposed in \cite{peng2011}.'' 
The language agrees with group $B1$.

In \cite{iqbalETal2016}, we read: ``There are many different proposals on how reproducible research can be guaranteed. These include approaches at reproducible practices, i.e., making other investigators able to repeat the process and calculations; re-analysis (as in the case of randomized trials); and replication by independent investigators, as in genetics, psychology, and cancer biology.''
This usage agrees with group $B1$.

\cite{stevens2017} includes explicit definitions: ``Replicability is `re-performing the experiment and collecting new data,' whereas reproducibility is `re-performing the same analysis with the same code using a different analyst.' Therefore, one can replicate a study or an effect (outcome of a study) but reproduce results (data analyses).''
This usage aligns with group $B1$.

We find in \cite{cartwright1991} the opposite terminology:
``I propose to distinguish replicability---doing the same experiment again---from reproducibility---doing a new experiment.'' 
This essay by a philosopher of science is a commentary on the work of another, who had proposed that ``replication is the establishment of a new and contested result by agreement over what counts as a correctly performed series of experiments''  \cite[]{collins1991}. 
Both authors are concerned with replication in economics, where usage generally does not distinguish between reproduce and replicate, and most often defaults to the latter word for everything. 
An even older example with no distinction of the two terms is \cite{dewaldETal1986}, while a more recent one is \cite{mccullough2008}. 
Like in political science, the economics literature uses `replication` much more frequently, often as an umbrella term. (A Google Scholar search with ``replication economics'' gives 298,000 results, and with ``reproducibility economics,'' 53,700 results; checked Jan. 19, 2018).
The new-section announcement for replication studies in the \emph{Journal of Applied Econometrics} \cite[]{pesaran2003}, for example, explains what is to be included: checking consistency and accuracy of data, checking validity of computations either directly or using different software (preferred), and checking ``if the substantive empirical finding of the paper can be replicated using data from other periods, countries, regions, or other entities as appropriate.'' 
Succinctly put by \cite{duvendackETal2017}: 
``we operationalize `replication' as any study whose main purpose is to determine the validity of one or more empirical results from a previously published study.''
%The recent terminology proposal of \cite{clemens2017} throw a wrench in this.

In psychological science, the widely cited works of the Open Science Collaboration---two large-scale efforts to replicate published empirical and correlational studies in psychology---use the terms reproduce and replicate interchangeably. 
Explicitly, the first work says: ``Some distinguish between `reproducibility' and `replicability' by treating the former as a narrower case of the latter (e.g., computational sciences) or vice versa (e.g., biological sciences). We ignore the distinction'.' \cite[]{openscience2012}. 
In the second work, the authors recurrently use `replication' (208 times), and less frequently apply `reproducible,' as an umbrella term (60 times). 
However, the lead author of these studies, Brian Nosek, adopts the distinction between terms according to Claerbout/Donoho/Peng when speaking in a recent video interview \cite[]{nsf2015vid}. 
He says:
``[by] reproducibility\ldots what we're usually referring to is: can you take the data and findings that I produced in some research, and run the analysis again and get the result back that I got.
[In] replicability\ldots I do a study, get a finding with some data, and you do your own study\ldots''





% ---------------------------------------------------------- END 


% ----------------------------------------------  REFERENCES 
{\small
\bibliography{repro}
\bibliographystyle{jponew}
}



%%%%%%%%
\end{document}
